\chapter{Architectures \& design patterns}

In this thesis, we will encounter a few architectures and design patterns. Before we do that, we should know what each of
them stands for. This chapter will go through the most notable patterns and describe them.

"Design Patterns are categorized mainly into three categories: Creational Design Pattern, Structural Design Pattern, and Behavioral
Design Pattern. These differ based on their level of detail, complexity, and scale of applicability to the entire system being designed." \cite{noauthor_classification_nodate}

\section{Creational design patterns}

As the name would suggest, Creational design patterns use abstraction for the instantiation process.
Using them makes a system independent of how its objects are created, composed, and represented. We can divide this pattern between
a class creational pattern, which uses inheritance to vary the class that's instantiated, and an object creational pattern, which will
delegate instantiation to another object. \cite{gamma_design_1995}

The main reason to use any of the creational design patterns is to help create an
abstraction layer for creating concrete instances of classes in a system.
This need comes with any system that is growing. Class inheritance may be enough at the start of the development,
but with a growing system comes a need to shift from class inheritance to delegation of instantiation to another object.
This is the time when creational design patterns come into play. These patterns give us much more flexibility
in what gets created, who creates it, how it gets created, and when.

\subsection{Abstract factory}

For an illustration, we will look at creating instances of different shapes.
We have got three shapes: triangle, square, pentagon. Each shape can be in different
variants like filled or with gradient background. The problem is how to abstract creating
each shape and its variant to match other objects of the same family. We also want to keep our
existing code intact, so we do not have to change our code repeatedly. \cite{gamma_design_1995}

To solve this problem with the abstract factory, we first declare interfaces for each distinct product of
the product family (e.g., triangle, square, pentagon). The next move is to declare the abstract factory, which is an interface with
a list of methods for all products that are part of the product family (for example, "createTriangle," "createSquare," "createPentagon").
These methods must return abstract types, which we declared at the beginning.

We create a separate factory class for product variants based on the abstract factory interface we declared for each of the variants.
We can call this implementation of the factory a concrete factory, which returns concrete products. In the client's code, we then accept the abstract
factory interface, giving us the ability to change the implementation of the concrete factory without the need to change the client's code.

One question is left unanswered, who creates the concrete factory instances? Usually, the application creates factories during
the initialization process; sometimes, these factories are part of the dependency injection container, but we will discuss this matter in another part of this chapter.

\subsection{Factory method}

Factory method provides an interface for creating objects in a superclass but allows subclasses to alter the type of created objects.
\cite{noauthor_factory_nodate} Using this solution, we will not use direct calls to the constructor in the application's code, but instead,
we will always call a defined method in our factory for a given class.

Imagine that we have an application that creates squares instances inside its code. As the application grows,
we encounter a new requirement to create another shape instead of a square. Each shape will have common methods.
If we continued to create objects inside the application's code, we would have to make a condition for each constructor invocation for shape.
Adding unnecessary complexity and this solution would be prone to errors.

Instead, we create a common interface for shapes. We create a base factory (abstract class) with logic shared between final shapes for this interface.
Concrete factory is then used to define which product (shape in our case) is returned to the application's code.

The client's code using this pattern does not see a difference between the concrete product returned by various concrete subclasses.
It only sees the abstract interface that we declared.

\subsection{Builder pattern}

In most applications, we encounter complex objects that require step-by-step initialization of many fields and nested objects leading to large and complex constructors which are hard to navigate through. These constructors then have many parameters and can be scattered throughout the application.

To illustrate this problem in the best way possible, imagine a room, a room can be a study room, dining room, kitchen, living room. Each room requires furniture and decoration, and some of it can be shared between rooms.

The first solution might be using subclasses, representing each room with different parameters to cover all combinations. Eventually, we will end up with a considerable number of subclasses. Any new parameter will require growing this hierarchy even more.

The builder pattern solves this issue by extracting each object's construction code from its class and moving it to separate objects called builders. Object construction is organized into a set of steps ("paintWalls", "buildTable", etc.). These steps need to be executed to create an object, resulting in the final product.

We can create several different builder classes that implement the same build steps with different internal implementations, giving us the ability to change the representation of the product. In our case, this could mean changing the material from which the furniture is made.

We can further extract a series of calls to the builder steps into a separate class called director. The director class defines the order in which to execute the building steps. \cite{noauthor_factory_nodate}

\subsection{Singleton}

Singleton pattern is used to ensure that a given class has only one instance during runtime. Using a singleton pattern solves two problems.

Ensures that a class has just a single instance. The most common reason someone would want to control how many instances are created is to control access to shared resources, like databases.

The second problem is providing a global access point to an instance of a class. This does sound like a global variable, but the singleton pattern solves the main problem with global variables, which is the ability to overwrite them accidentally.

The implementation of the singleton pattern should follow this implementation:
\begin{itemize}
    \item {In the class add a private static field that stores the singleton instance.}
    \item {Declare public static method to retrieve singleton instance, if an instance is not created yet, create it.}
    \item {Make constructor of the class private. The static method from the previous point should be the only one able to create a new instance of this class.}
\end{itemize}

\section{Structural patterns}

"Structural patterns are concerned with how classes and objects are composed to form larger structures. Patterns from this family help us compose interfaces or implementations. \cite{gamma_design_1995}

We will not go through all structural patterns like we did not go through all creational patterns in the previous section. If a reader wants to go through them all, an excellent online source is at https://refactoring.guru.

\subsection {Reflection}

The reflection pattern is used to access the class and its methods and fields. That means we can access the class and its methods and fields without knowing its implementation.
This has to be supported by the programming language itself. For example, C\# supports reflection and is widely used in some popular libraries.

Reflection works by scanning the program's implementations and creating metadata about the classes and methods. This metadata is stored in the program's memory and made accessible at
runtime.

