\chapter {Deployment}

This chapter will focus on deploying our library to the NuGet repository, GitHub actions, and setting up our repository to attract more contributors.

\section {NuGet repository}

The primary source of packages that are publicly available for .NET is \url{https://www.nuget.org}. Every developer can
upload their packages to this repository. Anyone can then search and download packages from this feed.

Before uploading our library to the NuGet repository, we need to set our project properly. We can find out
everything we need on the "Package authoring best practices" page in Microsoft's documentation
\url{https://docs.microsoft.com/en-gb/nuget/create-packages/package-authoring-best-practices}.

We know that we need to set the following properties from the documentation page.

\begin{itemize}
    \item \texttt{PackageId}: name of the package, which will be used in NuGet repository.
    \item \texttt{Authors}: list of authors of the package.
    \item \texttt{Description}: description of the package.
    \item \texttt{Copyright}: copyright details of the package.
\end{itemize}

With these properties set, we can now register on NuGet and create an \acrshort{api} key for uploading packages
using the command line. There is also a possibility to upload a package using the form on the NuGet page itself, but we want to
have this process automated using GitHub actions.

\section{GitHub actions}

GitHub actions is a tool for automation of \acrfull{cicd} process. This tool allows us to automate tasks like
running tests, checking build status, and publishing packages. It is also possible to run tasks on a schedule, like creating nightly
builds of our application or library.

We created three different workflows, each for a different operation in \acrshort{vcs}. The operations are:
\begin{itemize}
    \item push - every push of new commits onto the GitHub server will trigger this workflow.
    \item pull request - every new pull request to the main branch or update to the pull request will trigger this workflow.
    \item release - each new tag in the main branch will trigger this workflow.
\end{itemize}
These workflows will check our library's build status and the execution of the tests we wrote for our library.
In the case of release, the workflow will also create a new version of our library and publish it to the NuGet repository.

Workflows also offer us the possibility for static analysis of the codebase. This analysis is helpful as it scans code for common mistakes and
potential vulnerabilities. The static analysis is not used in this project, but it is something we can use in the future.

\section{Setting up repository}

One of our main goals with this library is that it will be easy for the community to expand our solution and report any issues with our library.
For this reason, we will follow the GitHub community standards. These standards should ensure that anyone can contribute to our project.
The list of community standards:
\begin{itemize}
    \item {Description}
    \item {README}
    \item {Code of conduct}
    \item {License}
    \item {Issue templates}
    \item {Pull request templates}
\end{itemize}

Some of these steps can be generated by GitHub, like license and code of conduct. Others have to be defined by us. In the repository
is a folder called \texttt{.github} which contains all the files that we need to set up, except for README and license files
which are in the root folder of the repository.

\section{Conclusion}

With the community standards set up, we have everything we need to release our library. Our repository on the GitHub server can be made public, and we
can release our library to the NuGet repository. We can now plan the next steps to finish our library.
