\chapter {Deployment}

In this chapter, will focus on deploying our library to NuGet repository, the use of GitHub actions
and also setting up our repository in a way, that will attract more contributors.

\section {NuGet repository}

The main source of packages that are publicly available for .NET, is \url{https://www.nuget.org}. Every developer can
upload their own packages to this repository. Anyone can then search and download packages from this feed.

Before we upload our library to the NuGet repository, we need to properly set our project. We can find out
everything we need on "Package authoring best practices" page in Microsoft's documentation
\url{https://docs.microsoft.com/en-gb/nuget/create-packages/package-authoring-best-practices}.

From documentation page, we know that we need to set following properties.

\begin{itemize}
    \item \texttt{PackageId}: name of the package, which will be used in NuGet repository.
    \item \texttt{Authors}: list of authors of the package.
    \item \texttt{Description}: description of the package.
    \item \texttt{Copyright}: copyright details of the package.
\end{itemize}

With these properties set, we can now register on NuGet and create an \acrshort{api} key for uploading package
using command line. There is also possibility to upload a package using form on NuGet page itself, but we want to
have this process automated using GitHub actions.

\section{GitHub actions}

GitHub actions is a tool for automation of \acrfull{cicd} process. This tool allows us to automate tasks like
running tests, checking build status, publishing packages, etc.





