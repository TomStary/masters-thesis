\chapter {State of the art C\# / .NET }

\section {C\#}
C\# is a general-purpose, type-safe, object-oriented programming language, the goal of which is programmer productivity.
To this end, the language balances simplicity, expressiveness,
and performance. \cite{albahari_c_2019}

Microsoft is developing and maintaining the C\# language. When writing this thesis, the current version of the C\# is C\#10.

The C\# code is statically compiled down to \acrlong{cil}. \acrshort{cil} cannot be run by itself on a machine.
\acrshort{cil} runtime or \acrfull{clr} must be used. Using \acrfull{jit} compilation, \acrshort{clr} reads \acrshort{cil} and translates \acrshort{cil} to native
code or sometimes called machine code. The machine's processor can then read machine code. Using \acrshort{cil} and \acrshort{clr} has benefits in running code
cross-platform without recompiling code for different processors, at the cost of some performance. \cite{rodenburg_code_2021}

\section {.NET}

.NET is a framework written for C\# and other languages such as F\# and
Visual Basic and Microsoft also develop them.
In their own words: ".NET is an open source developer
platform, created by Microsoft, for building many different types of applications." \cite{noauthor_what_nodate-2}

To compile a library or program with a .NET framework,
developers must first download and install a .NET \acrfull{sdk}.
.NET \acrfull{sdk} is either a standalone \acrfull{cli} tool or embedded inside an \acrshort{ide},
for example, in Visual Studio from Microsoft.