\chapter {State of the art EntityFramework and .NET platform}

If anyone wants to create \acrshort{orm} or \acrshort{ogm} for the .NET platform in \CS, they should go through some principles that are used in EntityFramework and .NET platform.
This chapter will go through these principles with insight into the technologies used.

\section {\CS}
\CS\ is a general-purpose, type-safe, object-oriented programming language, the goal of which is programmer productivity.
To this end, the language balances simplicity, expressiveness, and performance. \cite{albahari_c_2019}

Microsoft has created and is developing the \CS\ language. When writing this thesis, the current version of the \CS\ is \CS\ 10.

The \CS\ code is statically compiled down to \acrlong{cil}. \acrshort{cil} cannot be run by itself on a machine.
\acrshort{cil} runtime or \acrfull{clr} must be used.
Using \acrfull{jit} compilation, \acrshort{clr} reads \acrshort{cil} and translates \acrshort{cil} to native code or sometimes called machine code.
The machine's processor can then read machine code. Using \acrshort{cil} and \acrshort{clr} has benefits in running code cross-platform without recompiling code for different processors, at the cost of some performance. \cite{rodenburg_code_2021}

\section {.NET}

\textbf{.NET} is a framework written for \CS\ and other languages such as F\# and Visual Basic, which Microsoft also develops.
In their own words: ``.NET is an open-source developer platform, created by Microsoft, for building many different types of applications.'' \cite{microsoft_what_nodate}

The .NET platform went through many changes, and one of them was the introduction of the .NET Core version of the .NET platform.
.NET Core is a cross-platform implementation of the .NET platform. At the same time, the .NET Framework was also developed, which was only supported on the Windows platform.
The .NET Core and .NET Framework were merged in the .NET version 5.0.
From this version on, and the .NET platform has only one version for the whole platform.

To compile a library or program with a .NET platform, developers must first download and install a .NET \acrfull{sdk}.
.NET \acrfull{sdk} is either a standalone \acrfull{cli} tool or embedded inside an \acrshort{ide}, for example, in Visual Studio from Microsoft.

One part of the .NET platform we will need for any mapper will be a reflection.

\subsection{Reflection}

The reflection pattern is used to access the class and its methods and fields. We can access the class and its methods and fields without knowing its implementation.
This feature has to be supported by the programming language itself. For example, \CS\ supports reflection and is widely used in many popular libraries.

Reflection works by scanning the program's implementations and creating metadata about the classes and methods.
This metadata is stored in the program's memory and accessible at runtime.

For example, the following class:

\begin{listing}[H]
	\begin{minted}[
 frame=lines,
 framesep=2mm,
 baselinestretch=1,
 bgcolor=LightGray,
 linenos,
 breaklines
 ] {csharp}
 public class Person
 {
 [Key]
 public string Name { get; set; }
 public int Age { get; set; }
 }
\end{minted}
	\caption{Person class with a \texttt{Key} attribute}
\end{listing}

If we would want to know if the class does contain a field with KeyAttribute annotation, we could use the following code \ref{code:HasKeyAttribute} to get the MemberInfo instance:

\begin{listing}[H]
	\begin{minted}[
 frame=lines,
 framesep=2mm,
 baselinestretch=1,
 bgcolor=LightGray,
 linenos,
 breaklines
 ] {csharp}
 public bool HasKeyAttribute(Type type)
 {
 var members = type.GetMembers();
 return members.Any(member => member.GetCustomAttributes()
                                    .OfType<EndNodeAttribute>()
                                    .Any());
 }
 \end{minted}
	\caption{The example of using reflection}
	\label{code:HasKeyAttribute}
\end{listing}

We would use this code in cases where we do not know how an object is implemented.

\subsection {LINQ}

In the previous section, we used a method called \texttt{Any} to check if a collection contains an element. This method is part of a library in .NET called LINQ.
LINQ stands for Language Integrated Query, and it is a library that provides a set of methods that can be used to query objects.
We can filter, order, group, and transform data using this library.

LINQ is internally working as an expression tree, each command as an expression. The expression tree is immutable and evaluated at runtime, and developers
can use the tree to analyze and convert it to \acrshort{sql}, for example. To extend expression tree capabilities, LINQ provides an abstract class \texttt{ExpressionVisitor},
called for each expression combined with extension for \texttt{IQueryable<T>} and other tools.


With LINQ, we can write queries in a more readable way. These queries can then be translated to \acrshort{sql} queries, for example.
This feature is used in Entity Framework.

\section {Entity Framework Core}

The Entity Framework is an \acrshort{orm} with the support of writing queries using LINQ expressions.

When we are talking about Entity Framework, we are talking about the latest version of this framework, Entity Framework Core or EFCore, as it is known in the community.
This version was released for .NET Core 1.0, the first Microsoft version of .NET purposely built for multiplatform use.

To use Entity Framework, developers must add another dependency to their projects. This dependency is for Entity Framework and its called provider, which is used
to provide connection and extend translation capabilities for Entity Framework to work correctly over a specific database.

If we want to know how Entity Framework translates LINQ queries to SQL queries, we need to look at the implementation of the Entity Framework.
The translation is separated into several parts, but the most important parts are implementations of the \texttt{QueryableMethodTranslatingExpressionVisitor}\linebreak class which translates the LINQ method into custom expression, and the \texttt{QuerySqlGenerator} which translates the expression into SQL.
There are more expression visitors and generators, but these are the most important ones.

\subsection {\texttt{DbSet}}

An inseparable part of the Entity Framework Core is the \texttt{DbSet}. This class is used to query data from the database.
The Entity Framework creates it, and it is an implementation of \texttt{IQueryable<T>} interface.
This interface is the backbone of LINQ.

With this abstract class, we can create queries to the database and add or update entities to the change tracker.
Change tracker is then used for saving operation, where it is checked what changes in objects were made and then they are saved to the database using \texttt{DbContext.SaveChanges} method or its asynchronous version \texttt{DbContext.SaveChangesAsync}.

\texttt{DbSet} implements methods in both synchronous and asynchronous versions.
The main benefit of asynchronous versions is that they are not blocking the thread.
They are using \texttt{Task} to run the code on another thread from a thread pool or outside of the application's thread pool if possible.
This is done by using \texttt{async} keyword.

To show the exact process of generating a query from LINQ to SQL, we will go through the implementation of one of the methods in the \texttt{DbSet} class.

\subsection{\texttt{FindAsync}}

The \texttt{FindAsync} method is used to find an entity by its primary key. If the entity is not found, it returns \texttt{default} value for given object.
It takes 1 to n parameters of type \texttt{object}. These parameters are mapped to the primary keys of the entity.
If the number of parameters is not equal to the number of primary keys, the exception is thrown.

We are now going to analyze the implementation of the \texttt{FindAsync} method. The code for this method and the whole Entity Framework is available in the GitHub repository at \url{https://github.com/dotnet/efcore}.

Inside the concrete implementation of the \texttt{DbSet<T>} abstract class the\linebreak \texttt{FindAsync} method call method from \texttt{IEntityFinder} inteface called\linebreak\texttt{FindAsync}.
The \texttt{IEntityFinder.FindAsync} method is what starts the translation process of the query by creating the source LINQ query.
The find method is used the \texttt{FirstOrDefaultAsync} method, which is equivalent to \texttt{FirstOrDefault} method in LINQ. This method accepts a lambda function as a predicate for the query.
This predicate is build inside the \texttt{EntityFinder} class.

The \texttt{FirstOrDefaultAsync} is an extension method for \texttt{IQueryable} interface, and as the name suggests, it is used for retrieving the first element of a sequence or a default value if the sequence contains no elements.
This extension method starts the process of translation and enumeration of the result from the database.

Translation of the query works by overriding the default provider for\linebreak\texttt{IQueryable} done insede the \texttt{DbSet<T>} abstract class. The custom provider is defined by the \texttt{IAsyncQueryProvider} interface.
If we skip to the translation process itself, we will find ourselves in the \texttt{QueryCompilationCOntext} class inside, which is a method \texttt{CreateQueryExecutor}.
This method is responsible for transforming the expression tree for our query. The process is divided into a few steps preprocessing, translating methods, and postprocessing.
At the end of this method is the translation of the expression tree into a custom lambda function, which is used to execute the query, enumerate the result and return it.
Each step of the translation process uses an implementation of the \texttt{ExpressionVisitor} abstract class with different overrides.


\section {Summary}

We are now acquainted with Entity Framework and some of its principles. This chapter gave us an idea of what it means to implement
\acrshort{ogm} with LINQ support. Based on this research, we will build the part of the library responsible for translating LINQ to Cypher.