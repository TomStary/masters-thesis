\chapter {Proof-of-concept implementation}

\todo[inline]{introduction}

When writing this thesis, we will use the latest version of the C\# and .NET for implementation: C \# 10.0 and .NET 6.0.
The main tools used are Visual Studio Code with multiple plugins like Copilot and C\# extension. Github is used as VCS. Code is available at this URL https://github.com/TomStary/dotnet-neo4j-ogm or in the appendix of this thesis.

To create a new project, we will use the following command: \texttt{dotnet new classlib} with parameters for the name of the project and others.
We also want to separate tests from source code of library, so we employ a file structure like this:

\begin{figure}[H]
    \dirtree{%
        .1 dotnet-neo4j-ogm.
        .2 src.
        .3 {Neo4j.OGM}.
        .2 tests.
        .3 {Neo4j.OGM.}.
        .2 {.gitignore} .
        .2 {Neo4j.OGM.sln}.
    }
    \caption{File structure}
\end{figure}

With the file structure prepared, we can begin our implementation. During the development of this
library, we want to use \acrfull{tdd}. We will not go deep into this approach in this chapter
as it will be described in the next chapter, but keep in mind that during development, \acrshort{tdd} was used as it is an excellent way to write and test libraries.

\section {Session}

The session is the most significant part of this library. It is where we will contact the database to store and retrieve data.
For the session to work properly, we need to have a few things:

\begin{itemize}
    \item connection to the database
    \item domain metadata
\end{itemize}

Both the connection to a database and domain metadata can be acquired from the session's constructor or can be prepared elsewhere.

\todo[inline]{conclusion}


