\chapter {Proof-of-concept implementation}

For implementation we will use the latest version of the C\# and .NET at the time of writing this thesis, which are C\# 10.0 and .NET 6.0.
Other used tools are Visual Studio Code with multiple plugins like Copilot and C\# extension. Github is used as VCS and code can be
found at this URL https://github.com/TomStary/dotnet-neo4j-ogm or in appendix of this thesis.

To create a new project, we will use the following command: \texttt{dotnet new classlib} with parameters for the name of the project and others.
We also want to separate tests from source code of library, so we employ a file structure like this:

\begin{figure}[H]
    \dirtree{%
        .1 dotnet-neo4j-ogm.
        .2 src.
        .3 {Neo4j.OGM}.
        .2 tests.
        .3 {Neo4j.OGM.}.
        .2 {.gitignore} .
        .2 {Neo4j.OGM.sln}.
    }
    \caption{File structure}
\end{figure}

With file structure prepared we can begin our implementation. During development of this
library we want to use \acrfull{tdd}. We will not go deep into this approach in this chapter
as it will have its own chapter, but keep in mind, that during development, \acrshort{tdd} was used as it is a good way to write and test libraries.

\section {Session}

Session is the most significant part of this library. It is the place where we will contact database to store and retrieve data.
For session to work properly, we need to have a few things:

\begin{itemize}
    \item connection to the database
    \item domain metadata
\end{itemize}

Both connection to database and domain metadata can be aquired from the constructor of the session or can be prepared elsewhere.

