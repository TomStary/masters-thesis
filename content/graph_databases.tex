\chapter {State of the art of graph databases}

The first thing we need to know about graph databases is their definition:
"Graph databases store information in graphs, very similarly as relational databases store information in tables and have relations between columns in tables, graph databases store information in nodes and even on the edges, or as we should call them, relations." \cite{morgante_what_2021}

To better show the power of graph databases, we will show it in a real-world example.
Social networks dominate today's internet content, and everyone wants to be connected with his or her friends and relatives.
In one of the social networks, we have people who can post their thoughts or opinions.
Other users can follow them to see their posts and like them.
We can identify two entities with relations between them.
In a relational database, it would look like this.

\begin{figure}[H]
	\centering
	\includegraphics[width=0.5\textwidth]{content/thesis-db.png}
	\caption{Relational database for social network}
	\label{fig:relscheme}
\end{figure}

From the schema in figure \ref{fig:relscheme}, we can see that we need four tables and five foreign keys to describe all relations between two identified entities.
This design can lead to costly joins to answer requests resulting from relationships between entities.
For example, we could ask who follows followers who liked someone's post.
The query for this example would look like the example in code \ref{code:followers_of_followers_liked}.

\begin{listing}[H]
	\begin{minted}
 [
 frame=lines,
 framesep=2mm,
 baselinestretch=1,
 bgcolor=LightGray,
 linenos,
 breaklines
 ] {sql}
SELECT followers_of_followers_data.*
FROM post
JOIN user_likes fl ON post.id = fl.id_post
JOIN "user" followers_liked ON fl.id_follower = followers_liked.id
JOIN user_follows followers_of_followers ON followers_liked.id = followers_of_followers.id_following
JOIN "user" followers_of_followers_data ON followers_of_followers.id_follower = followers_of_followers_data.id
WHERE post.id = :post_id
ORDER BY followers_of_followers_data.name;
 \end{minted}
	\caption{SQL query for getting followers of users who liked a post}
	\label{code:followers_of_followers_liked}
\end{listing}

The query in code \ref{code:followers_of_followers_liked} is complex and hard to read. However, the main problem is performance.
As data in the database grows, the query execution time will be longer due to the increasing size of the data needed to load.
This increase in execution time will happen regardless of the change in the actual result.

Now, we compare the same problem solved using a graph database.

\begin{figure}[H]
	\centering
	\includegraphics[width=0.8\textwidth]{content/graph_example.png}
	\caption{Graph database for social network}
	\label{fig:graphscheme}
\end{figure}

The figure \label{fig:graphscheme} is the schema from the Neo4j database.
Each node has one relation or more to other nodes.
Furthermore, the answer to our question from the beginning is already visible. The question was: who follows users that liked the post?
To get the answer, we can follow the path in the graph, and as we are going to discover in a moment, the query to get this information also follows this path.

\begin{listing}[H]
	\begin{minted}
 [
 frame=lines,
 framesep=2mm,
 baselinestretch=1,
 bgcolor=LightGray,
 linenos,
 breaklines
 ] {cypher}
 MATCH (:Post {id: {post_id}})-[:LIKES]->(:User)-[:FOLLOWS]->(followers:User) RETURN (followers);
 \end{minted}
	\caption{Cypher query for getting followers of followers who liked a post}
	\label{code:cypherQuery}
\end{listing}

As promised, the query in the example \ref{code:cypherQuery} should not be hard to understand.

This small example was inspired by a blog post from Graham Cox, Introduction to Graph Databases. \cite{cox_introduction_2017}
It is designed to show the strength of graph databases compared to relational databases.

\section {Native vs Non-native graph databases}

When dealing with graph databases, we have to look at two main \acrshort{dbms} features: storage and processing.

Storage is how graphs are stored in memory. If the storage is optimized for graphs, like having related nodes close together, we talk about native graph storage.
When implementation uses other NoSQL storage, they are called non-native graph storage.

The second feature is processing, which refers to how graph databases process database operations.
What is meant by that is how the database treats queries and how it handles storage.
Native databases use Index-free adjacency for processing.
\cite{chao_graph_2018}

\textbf{Index-free adjacency}: "Native graphs take data that is logically connected via arcs or relationships and hard-wire the physical RAM addresses of these items into the node." \cite{mccreary_neighborhood_2021}
This information should answer to why are graph databases faster than other types of \acrshort{dbms} in searching related row, nodes or documents.
In traditional \acrshort{rdbms}, looking up a row in another table means pulling an index table representing this relation and then finding a path to the row in said table.
This behaviour leads to another problem in \acrshort{rdbms} because the database uses many indexes to keep data connected, which negatively impacts insert operations.

One more thing we should go through is how the graph database handles writes. Connected data requires strict data integrity.
Graph databases have to create or update nodes and relationships in one transaction; otherwise, this could result in a corrupted graph, which is almost impossible to fix.
The solution to this problem is to write fully \acrshort{acid}-compliant transactions, ensuring that the database will not become corrupted.
\cite{chao_graph_2018}

\section{Neo4j}

Neo4j is a graph database management system developed by Neo4j, Inc. \cite{neo4j_company_nodate}
Neo4j is a native graph database and \acrshort{acid}-compliant.
Besides \acrshort{dbms}, the Neo4j company created a custom query language for graph databases called Cypher. \cite{neo4j_neo4j_nodate}

\section{Cypher}

\textbf{Cypher} is a query language created by Neo4j initially for their graph database.
Nowadays, it is possible to use Cypher on other graph databases using openCypher, an open-source project for other graph \acrshort{dbms}. \cite{noauthor_resources_nodate}

From Neo4j's documentation: Its philosophy is to be easily read and understood by developers, database professionals, and business stakeholders.
Its ease of use derives from the fact that it is in accord with how we intuitively describe graphs using diagrams. \cite{robinson_graph_2015}

\subsection{Nodes}

To depict nodes in Cypher, we surround the node with parentheses, e.g., \texttt{(node)}.
Parentheses were chosen because they look like circles, a standard visual representation of nodes in the graph. \cite{noauthor_getting_nodate}

If we need to refer to the node, we can give it a variable like \texttt{(u)} for a user or \texttt{(p)} for a post.
In real-world queries, full names of variables should be used to understand the query better.

With variables, we mentioned the possibility of giving a variable name to the nodes, but how can we distinguish two nodes from one to the other.
We can do this by assigning labels to each node. Labels are like tags or table names, which specify certain entities in the graph.
If we look back to our example of users and posts, we already used the two labels: \texttt{(p:Post)} and \texttt{(u:User)}.

Specifying labels also has another benefit. When not using labels in a query, the database has to look for all nodes, which can negatively impact the performance of the query.
\cite{noauthor_getting_nodate}

\subsection{Relationships}

Relationships are representations of edges in a graph between nodes.
They are marked in Cypher using an arrow \texttt{-->} or \texttt{<--} between two nodes.
Additional information, such as by which relationship type are two nodes connected and any properties of the relationship, can be placed in square brackets inside the arrow \texttt{((p:Post)-[:AUTHOR]->(u:User))}. \cite{noauthor_getting_nodate}

The direction of the relationship must be present only while creating the relationship.
During the traversal of the graph, it is possible to omit the direction by using two dashes (\texttt{--}).
This syntax can make queries more flexible and not force users to know in which directions are relationships stored in the database.
The tradeoff is a small performance loss.

Like with nodes, variables can be used to refer to relationships.
If we do not need to reference the relationship later. We can leave any specification for a relationship using two dashes (\texttt{--}).

\subsection{Nodes and relationship properties}

One thing that was not mentioned yet regarding nodes and relationships is properties. Each node and even each relationship can have one or more properties assigned to them.

Properties are name-value pairs providing additional detail. To represent them in the query, we place them in curly brackets. \cite{noauthor_getting_nodate}
Below is two examples of this usage, one for node and the other for relationship.

\begin{itemize}
	\item {Node property: \texttt{(u:User \{ nickname: 'mr. Incognito' \} )}}
	\item {Relationship property: \\*\texttt{-[rel:AUTHOR \{ posted: 2022-02-05T12:12:20Z \} ]}}
\end{itemize}

\subsection{Querying with Cypher}

Cypher has few words reserved for specific actions called keywords like most other programming languages. \cite{neo4j_querying_nodate}
First, look at the two most common keywords:
\begin{itemize}
	\item {\texttt{MATCH}: This keyword is what searches for an existing node, relationship, label, property, or pattern in the database.
	      \texttt{MATCH} does work similarly to the \texttt{SELECT} in \acrshort{sql}.}
	\item {\texttt{RETURN}: The \texttt{RETURN} keyword defines what values or results we want to retrieve from the database.
	      It is used mainly in search queries as it is not required to be used during writing procedures.
	      \texttt{RETURN} does utilize the node and relationship variables. If we want to return any results from defined \texttt{MATCH},
	      we must specify which nodes, relationships, properties, or patterns we want to return.}
\end{itemize}

\subsection{Create, update, and delete operations}

Besides queries, a proper database system must have methods to create, update or delete data.

The function \texttt{CREATE} is used to insert new data into a database.
Using \texttt{CREATE}, we can create nodes, relationships, and also patterns.
Below are some examples (\ref{code:create_user}, \ref{code:create_relationship}) of how \texttt{CREATE} is used. \cite{neo4j_updating_nodate}

\begin{listing}[H]
	\begin{minted}
 [
 frame=lines,
 framesep=2mm,
 baselinestretch=1,
 bgcolor=LightGray,
 linenos,
 breaklines
 ] {cypher}
 CREATE (u:User {nickname: "mr. Incognito"})
 RETURN u
 \end{minted}
	\caption{Create a new user with a nickname}
	\label{code:create_user}
\end{listing}

In the following example \ref{code:create_relationship}, we will use \texttt{MATCH} to create a relationship between two nodes.
If we used the \texttt{CREATE} keyword for creating the relationship without the \texttt{MATCH}, we would introduce duplicities of both nodes.

\begin{listing}[H]
	\begin{minted}
 [
 frame=lines,
 framesep=2mm,
 baselinestretch=1,
 bgcolor=LightGray,
 linenos,
 breaklines
 ] {cypher}
 MATCH (u:User {nickname: "mr. Incognito"})
 MATCH (p:Post {title: "Hello, world!"})
 CREATE (u)<-[:AUTHOR]-(p)
 \end{minted}
	\caption{Create a new relationship between two nodes}
	\label{code:create_relationship}
\end{listing}

There is another way to create this relationship, and we will look at it later in this chapter.

To update data in the database, Cypher uses the \texttt{SET} keyword, which can create or update node or relationship properties.

If we want to delete a node or relationship, we use the \texttt{DELETE} keyword.
This is similar to how \acrshort{sql} \texttt{DELETE} works, but with one exception. If a node is in a relationship
with another node, we cannot delete it because it would create an inconsistent graph, with a potential relationship pointing to nothing. \cite{neo4j_updating_nodate}
We could run two queries to delete the relationship and delete the node itself, but there is a more straightforward solution.
We can use \texttt{DETACH DELETE}, which does detach all relationships from the node before deleting it.

\section{\acrshort{orm}}
\acrshort{orm} stands for an object-relational mapper based on the object-relational mapping concept.
Object-relational mapping is the idea of writing queries using the object-oriented paradigm.
There are some limitations to what \acrshort{orm} can accomplish.
Developers should always consider these limits before using an \acrshort{orm} framework. \cite{hoyos_what_2018}

\noindent Pros:
\begin{itemize}
	\item There is no need to use a second language during software development, \acrshort{sql} is a powerful language, but most developers do not use it too often.
	\item \acrshort{orm} abstracts away from the database system.
	\item It can lead to better performance than writing queries by ourselves.
\end{itemize}
Cons:
\begin{itemize}
	\item If a developer is an \acrshort{sql} power user, he can write queries that will have better performance.
	\item Developers have to learn how to use \acrshort{orm} properly.
	\item Developers still need to know how \acrshort{orm} works under the hood.
\end{itemize}

Using the term \acrshort{orm} with graph databases is not correct. The proper term would be \acrshort{ogm} (object-graph mapper),
but there are a few reasons why we are using \acrshort{orm} instead of \acrshort{ogm} in the name of this thesis.
However, please make no mistake. When discussing \acrshort{orm} involving graph databases, it is, in fact, \acrshort{ogm}.
The main reason is simple: \acrshort{orm} has been around for more than a decade, and developers are familiar with the concept and its challenges.

The differences between \acrshort{orm} and \acrshort{ogm} are mainly in the target \acrshort{dbms}.
They are not interchangeable, however.

\section{Summary}

This chapter introduced graph databases and compared them to relational \acrshort{dbms}.
We compared the queries of both types of databases and their differences.
We also studied the differences between native and non-native databases.

The rest of this chapter focused on Neo4j and Cypher language, where we introduced the basics of this language, like how relationships and nodes are defined and base keywords used in Cypher.

In the end, we also adequately introduced the concept of \acrshort{orm} and its relation to \acrshort{ogm}.
