\chapter {Similar solutions}

\section{Neo4j-\acrshort{ogm}}

Neo4j company has already created an \acrshort{ogm} for their database. It supports dynamic objects and maps nodes and their relations into the domain model.
It is written in Java, but it should be a starting point when designing a similar library in the C\#.

\noindent Main features:
\begin{itemize}
    \item Object graph mapping of annotated node- and relationship-entities
    \item Neo4jSession for direct interaction with Neo4j
    \item Fast class metadata scanning
    \item Optimized management of data loading and change tracking for minimal data transfers
    \item Multiple transports: binary (bolt), \acrshort{http} and embedded
    \item Persistence lifecycle events
    \item Query result projection to data transfer objects (\acrshort{dto})
\end{itemize}

The following sections are information obtained from the official documentation of Neo4j-OGM for Java. \cite{noauthor_reference_nodate}

\subsection {Neo4j drivers}

There are three possible drivers to use, Bolt driver, \acrshort{http} driver, or embedded driver, which creates in-memory Neo4j instance.

Using a different driver in the development or test environment will not affect the production code. These drivers are interchangeable without the need for modification in queries.

\subsection {Entities}

This library offers us the possibility to define and shape entities and relationships. \texttt{@NodeEntity} annotation is used to declare that a \acrfull{pojo}
is indeed a node. This class must have one empty public constructor to allow the library to construct the objects.

Fields on the entity are by default mapped to properties of the node. Fields referencing other node entities (or collections thereof) are linked with relationships.

If we want to change fields name or other properties, we can use annotations like \texttt{@Property}, \texttt{@Id}, \texttt{@GeneratedValue}, or \texttt{@Relationship}. On the other hand,
if we want to not include a field in the node, we can use annotations \texttt{@Transient}.

\subsection {Relationships}

Every field of an entity that references one or more other node entities is backed by relationships in the graph. These relationships are managed by Neo4j-\acrshort{ogm} automatically.

If we want to specify relationship properties, like the direction of the relationship, the \texttt{@Relationship} annotation is used. The directions are either \texttt{INCOMING},
\texttt{OUTGOING}, or \texttt{UNDIRECTED}, where the last one ensures that the path between two node entities is navigable from either side.

Neo4j gives us the ability to define properties in relationships. Neo4j solves this using an entity (\acrshort{pojo}) with annotation \texttt{@RelationshipEntity}.

A String attribute called type is available on the \texttt{@RelationshipEntity} annotation to control the relationship type. Like the simple strategy for labeling node entities,
if this is not provided, then the class's name is used to derive the relationship type, although it is converted into \texttt{SNAKE\_CASE} to honor
the naming conventions of Neo4j relationships. \cite{noauthor_reference_nodate}

Inside the entity, we then define @StartNode and \texttt{@EndNode}. In referenced entities, we also define a reference to
the related entity and use @Relationship with type same as in \texttt{@RelationshipEntity}.

Below is an example of such usage:

\begin{listing}[!ht]
    \begin{minted}
    [
    frame=lines,
    framesep=2mm,
    baselinestretch=1.2,
    bgcolor=LightGray,
    linenos
    ] {java}
@NodeEntity
public class Actor {
    Long id;
    @Relationship(type="PLAYED_IN") private Role playedIn;
}

@RelationshipEntity(type = "PLAYED_IN")
public class Role {
    @Id @GeneratedValue   private Long relationshipId;
    @Property  private String title;
    @StartNode private Actor actor;
    @EndNode   private Movie movie;
}

@NodeEntity
public class Movie {
    private Long id;
    private String title;
}
    \end{minted}
    \caption{An example of entities with relationship as separate entity}
    \label{listing:1}
\end{listing}

\subsection {Indexes}

Annotations can define indexes. We already saw one of them, \texttt{@Id}, an annotation for the primary index.

But primary indexes are not the only type of index we can define in our models. We can also define indexes for other
properties using \texttt{@Index} annotation. We get an index with constraint if we use \texttt{@Index(unique=true)}.

This library also supports composite indexes and node constraints with \texttt{@CompositeIndex} and \texttt{@CompositeIndex(unique = true)}, respectively.

\texttt{@Required} is an existence constraint. "It is possible to annotate properties in both node entities and relationship entities. For node entities
the label of declaring class is used to create the constraint. For relationship entities the relationship type is used - such type must
be defined on leaf class." \cite{noauthor_reference_nodate}

The \acrshort{ogm} library can handle creating and managing indexes or constraints, but as stated in the documentation, this feature should be used only for development
and not in production. That is why this feature is, by default, turned off.


These are the available modes:
\begin{table}[H]
    \begin{center}
        \begin{tabularx}{\textwidth}{|c|p{0.83\textwidth}|}
            \hline
            node     & Default, nothing is done on the side of the OGM library.                                  \\
            \hline
            validate & This ensures that all constraints and indexes are in the database before starting up.     \\
            \hline
            assert   & This drops all indexes on startup and then creates only these defined by OGM annotations. \\
            \hline
            update   & Update indexes and constraints based on annotations.                                      \\
            \hline
            dump     & Dumps all indexes and constraints to a file.                                              \\
            \hline
        \end{tabularx}
        \caption{Available modes for indexes and constraints}
    \end{center}
\end{table}

\subsection{Sessions}
To interact with mapped entities, Neo4j-OGM requires a \texttt{Session}, which \texttt{SessionFactory} provides. Besides providing \texttt{Session}, \texttt{SessionFactory}
also setups up the object graph mapping metadata when constructed. That is then used across all \texttt{Session} objects created by \texttt{SessionFactory}.

Session keeps track of mapped entities, their changes, and changes between their relationships. Tracking is then used when saving or otherwise
working with mapped entities. When an entity is loaded by session, reloading this entity is then done from cache within the session.

To keep new data and not prolong sessions too much, session lifetime can be managed in code. Too long session lifetime
means that other users can change data, and too short a lifetime means costly save operations will be executed more often. There is a way
to force the session's cache to clear, but it is advised against it.

Cypher drives Neo4j-\acrshort{ogm} queries only. This means that only \acrfull{crud} operations are available. Documentation
suggests using server-side operations for more complex or performant graph traversals over the graph.
Nevertheless, Cypher should be powerful enough for most of the problems.

\subsection {Persisting entities}

Session allows to save, load, loadAll, and delete entities with transaction handling, and exception translation managed.
Persistence is performed through method save. This method then looks at underlying \texttt{MappingContext} and compares data loaded
from the database with the saved entity, creating appropriate Cypher queries to update the database based on differences.
Calling save is necessary to propagate changes because Neo4j-\acrshort{ogm} does not automatically commit changes.

The method has a second optional parameter: the depth, which can restrict how much depth the save will perform.
The default value is \texttt{-1}, which means saving every change in node and all reachable nodes from it into the database.
This approach is recommended because of possible inconsistencies that could happen.

\subsection {Loading entities}

Loading entities can be done using methods \texttt{session.loadxxx} or writing a custom Cypher query with methods \texttt{session.query} and \texttt{session.queryForObject}.
Like the depth for saving function, the load functions also have a depth.

Depth is there to determine how many depths of relatives will be loaded with query. The default behavior is to load the object's simple properties and neighbors.
This represents loading data using depth set to value 1. Depth is mainly helpful when loading deeper than broader parts of a graph. Depth also helps developers to
execute fewer load operations from the database.

When using load methods from the session, the session uses \texttt{LoadStrategy} to generate a \texttt{RETURN} clause. The default strategy is schema loading,
which uses entities metadata. The other is the path load strategy that uses paths from the root node. It is possible to change the strategy for a query using \texttt{Session.setStrategy}
or globally by calling \texttt{SessionFactory.setStrategy}.

\subsection{Transactions}

Neo4j uses transactions, which means queries can be executed only in transaction boundaries. Neo4j-\acrshort{ogm}
offers tools to manage transactions, but the developer does not have to use them because the session handles them independently but with the auto-commit transaction.
