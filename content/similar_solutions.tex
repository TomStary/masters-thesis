\chapter {Existing OGM libraries for graph databases}

Searching for \acrshort{ogm} for Neo4j in \CS\ did not result in any working, well maintained, and well documented \acrshort{ogm} library.
However, there is a library written by the Neo4j company itself in Java called Neo4j-\acrshort{ogm}.
In this chapter, we will study this library and conclude its relevance for our goal of building a \acrshort{ogm} library for the .NET framework.

\section{Neo4j-\acrshort{ogm}}

Neo4j company has already created an \acrshort{ogm} for their \acrshort{dbms}.
It supports dynamic objects and maps nodes and their relations into the domain model written in Java.

The list of features from the official documentation: \cite{neo4j_neo4j_nodate}
\begin{itemize}
	\item Object graph mapping of annotated node- and relationship-entities
	\item Neo4jSession for direct interaction with Neo4j
	\item Fast class metadata scanning
	\item Optimized management of data loading and change tracking for minimal data transfers
	\item Multiple transports: binary (bolt), \acrshort{http} and embedded
	\item Persistence lifecycle events
	\item Query result projection to data transfer objects (\acrshort{dto})
\end{itemize}

The following sections are information obtained from the official documentation of Neo4j-\acrshort{ogm} for Java. \cite{neo4j_reference_nodate}

\subsection {Neo4j drivers}

There are three possible drivers to use, Bolt driver, \acrshort{http} driver, or embedded driver, which creates an in-memory Neo4j database instance.

Using a different driver in the development or test environment will not affect the production code. These drivers are interchangeable without the need for modification in queries.

\subsection {Entities}

The library offers the possibility to define and shape entities and relationships.
\texttt{@NodeEntity} annotation is used to declare that a \acrfull{pojo}is a representation of a node.
This class must have one empty public constructor to allow the library to construct the objects.

Fields on the entity are by default mapped to properties of the node. Fields referencing other node entities (or collections) are linked with relationships.

If we want to change fields name or other properties, we can use annotations like \texttt{@Property}, \texttt{@Id}, \texttt{@GeneratedValue}, or \texttt{@Relationship}.
On the other hand, if we want to not include a field in the node, we can use \texttt{@Transient} annotation.

\subsection {Relationships}

Every field of an entity that references one or more other node entities is backed by relationships in the graph. These relationships are managed by Neo4j-\acrshort{ogm} automatically.

If we want to specify relationship properties, like the direction of the relationship, the \texttt{@Relationship} annotation is used.
The directions are either \texttt{INCOMING}, \texttt{OUTGOING}, or \texttt{UNDIRECTED}, where the last one ensures that the path between two node entities is navigable from either side.

Relationships in a graph database can have properties assigned to them. Neo4j-\acrshort{ogm} supports this feature using an (\acrshort{pojo}) with annotation\linebreak\texttt{@RelationshipEntity}.

A String attribute called "type" is available on the \texttt{@RelationshipEntity} annotation to control the relationship type.
Like the simple strategy for labelling node entities, if "type" is not provided, the class's name derives from the relationship type, although it is converted into a snake case with an upper casing to honour the naming conventions Neo4j relationships. \cite{neo4j_reference_nodate}

Inside the entity, we then define \texttt{@StartNode} and \texttt{@EndNode} annotations.
In referenced entities, we also define a reference to the related entity and use \texttt{@Relationship} annotation with the same type as is in \texttt{@RelationshipEntity} annotation.

In the code example \ref{listing:1} is a simple example of using \texttt{@RelationshipEntity}.

\begin{listing}[!ht]
	\begin{minted}
 [
 frame=lines,
 framesep=2mm,
 baselinestretch=1,
 bgcolor=LightGray,
 linenos
 ] {java}
@NodeEntity
public class Actor {
 Long id;
 @Relationship(type="PLAYED_IN") private Role playedIn;
}

@RelationshipEntity(type = "PLAYED_IN")
public class Role {
 @Id @GeneratedValue private Long relationshipId;
 @Property private String title;
 @StartNode private Actor actor;
 @EndNode private Movie movie;
}

@NodeEntity
public class Movie {
 private Long id;
 private String title;
}
 \end{minted}
	\caption{An example of model with relationship entity}
	\label{listing:1}
\end{listing}

\subsection {Indexes}

Indexes are also defined by using an annotation. We already saw one of them: the \texttt{@Id} annotation used for the primary index.

Primary indexes are not the only type of index we can define in our model.
We can also define indexes for other properties using \texttt{@Index} annotation, and the index will have unique constraint if we use \texttt{@Index(unique=true)}.

This library also supports composite indexes and node constraints\linebreak with \texttt{@CompositeIndex} and \texttt{@CompositeIndex(unique = true)} annotations, respectively.

\texttt{@Required} is an existence constraint. "It is possible to annotate properties in both node and relationship entities.
For node entities the label of declaring class is used to create the constraint.
For relationship entities the relationship type is used - such type must be defined on leaf class." \cite{neo4j_reference_nodate}

The library can handle creating and managing indexes or constraints, but as stated in the documentation \cite{neo4j_reference_nodate}, this feature should be used only for development and not in production. That is why this feature is, by default, turned off.

These are the available modes for managing indexes and constraints:
\begin{table}[H]
	\begin{center}
		\begin{tabularx}{\textwidth}{|c|p{0.83\textwidth}|}
			\hline
			node     & Default, nothing is done on the side of the OGM library.                                  \\
			\hline
			validate & This ensures that all constraints and indexes are in the database before starting up.     \\
			\hline
			assert   & This drops all indexes on startup and then creates only these defined by OGM annotations. \\
			\hline
			update   & Update indexes and constraints based on annotations.                                      \\
			\hline
			dump     & Dumps all indexes and constraints to a file.                                              \\
			\hline
		\end{tabularx}
		\caption{Available modes for indexes and constraints}
	\end{center}
\end{table}

\subsection{Sessions}
To interact with mapped entities, Neo4j-\acrshort{ogm} requires an instance of the \texttt{Session} class, which can be created by using \texttt{SessionFactory} class.
Besides creating \texttt{Session}, \texttt{SessionFactory} also setups up the object graph mapping metadata when constructed.
The metadata are used across all \texttt{Session} instances created by \texttt{SessionFactory}.

Session keeps track of mapped entities, their changes, and changes in their relationships.
Tracking is then used when saving or otherwise working with mapped entities.
When an entity is loaded by an instance of the \texttt{Session} class, the result is cached within the \texttt{Session} instance.

To keep new data and not prolong sessions too much, session lifetime can be managed in code.
Too long session lifetime means that other users can change data causing concurrency exceptions, and too short a lifetime means costly save operations will be executed more often.
There is a way to force the session's cache to clear, but it is advised against it.

Neo4j-\acrshort{ogm} use Cypher queries only for its operations, which limits the capabilities of the Neo4j-\acrshort{ogm} library.
Documentation suggests using server-side operations for more complex or performant graph traversals over the graph.
Nevertheless, Cypher should be powerful enough for most of the problems.

\subsection {Persisting entities}

The session allows to save, load and delete entities with transaction handling and exception translation managed.
Persistence is performed through method save. This method then looks at underlying \texttt{MappingContext} and compares data loaded
from the database with the saved entity, creating appropriate Cypher queries to update the database based on differences.
Calling save is necessary to propagate changes because Neo4j-\acrshort{ogm} does not automatically commit changes.

The save method has a second optional parameter: the depth, which can restrict how deep will the save operation in the graph for given entities.
The default value is \texttt{-1}, which means saving every change in node and all reachable nodes from it into the database.
This approach is recommended because of possible inconsistencies that could happen.

\subsection {Loading entities}

Loading entities can be done using methods \texttt{session.loadxxx} or writing a custom Cypher query with methods \texttt{session.query}\linebreak and \texttt{session.queryForObject}.
Like the depth parameter for the saving function, the load functions also have a depth parameter.

Depth is there to determine how many depths of relatives will be loaded with the query.
The default behaviour is to load the object's properties and neighbours.
This behaviour represents loading data using a depth set to value 1.
Depth is mainly helpful when loading deeper than broader parts of a graph and helps developers execute fewer load operations from the database.

When using load methods from the session, the session uses \texttt{LoadStrategy} to generate a \texttt{RETURN} clause. The default strategy is schema loading,
which uses entities metadata. The other is the path load strategy that uses paths from the root node. It is possible to change the strategy for a query using \texttt{Session.setStrategy}
or globally by calling \texttt{SessionFactory.setStrategy}.

\subsection{Transactions}

Neo4j uses transactions, which means queries can be executed only in transaction boundaries. Neo4j-\acrshort{ogm}
offers tools to manage transactions. The developer does not have to use them because the session handles them independently. However, with the auto-commit transaction.

\section{Summary}

This chapter introduced the Neo4j-\acrshort{ogm} library, its features, and how to use them. We studied them to understand what is needed for
an implementation of an \acrshort{ogm} in any language.